\documentclass{article}	
\usepackage{graphicx}

\usepackage{amsmath}
\usepackage{amsmath}
\usepackage{graphicx}
\usepackage{tabularx}
\usepackage{float}

\begin{document}
% Title Page
\title{Why Linux Is Better!}
\author{Arnav Prasad , MIS : 612203011}
\date{\today}
\maketitle

\pagenumbering{gobble}
\newpage
\pagenumbering{roman}
\tableofcontents
\newpage


\newpage

\begin{figure}
\includegraphics[width=0.7\textwidth]{file-12-dtl.jpeg}
\caption{Why Linux Is The Best!}
\label{fig:Why Linux is the best}
\end{figure}


%  Page 2
\pagenumbering{arabic}
\section{OVERVIEW}

Linux is an open-source operating system kernel that serves as the core foundation for various Linux distributions (like Ubuntu, CentOS, and Debian). It provides the essential software components and services necessary to manage hardware resources and run software applications on a computer. 
\paragraph{}
Linux is known for its stability, security, and flexibility and is widely used in servers, desktop computers, embedded systems, and more.

\subsection{Open Source and Free}
 Most Linux distributions are open source, which means their source code is freely available. This allows for transparency, collaboration, and customization. Additionally, Linux distributions are often available for free, which can be appealing for individuals and organizations with budget constraints.

\subparagraph{Open Source Community}
The vibrant open-source community continuously contributes to improving Linux's performance. With collaboration and testing from developers worldwide, Linux remains at the forefront of performance optimization.

\subparagraph{File Managers}
Linux file managers often allow for customization of file handling, views, and shortcuts, improving file organization and navigation.


\paragraph{Stability and  Reliability}
Linux is known for its stability and reliability. Many servers and critical infrastructure systems run on Linux due to its ability to run for extended periods without requiring frequent reboots.

\subsection{Security}
 Linux's architecture and security model contribute to its reputation for being more secure than some other operating systems. The community's quick response to security vulnerabilities and the ability to fine-tune security settings are considered advantages.


\subsection{Community and Support}
 The Linux community is active and passionate, offering extensive online resources, forums, and documentation. Users can easily find help and solutions to their problems.

\paragraph{Privacy}
Linux distributions generally prioritize user privacy and data protection. Users have more control over what data is collected and shared.

\subparagraph{Licensing}
The licensing of Linux and open-source software in general allows users to modify and distribute the software without the same restrictions as proprietary software.



\subsection{Customizability}
Linux offers a high degree of customization. Users can choose from a variety of desktop environments, themes, and software components to create a personalized computing environment that suits their needs.

\paragraph{Compatibility and Portability}
Linux is highly portable and can run on various hardware architectures. It's commonly used in embedded systems, mobile devices, and IoT (Internet of Things) devices.

\subparagraph{}
Linux Mint offers various C++ Integrated Development Environments (IDEs) to choose from, and you can install them based on your preferences. 

\paragraph{CUSTOMIZATIONS}
Linux offers extensive customization options that empower users to personalize their computing experience. You can start by choosing from a variety of desktop environments, each with its own look and feel, or opt for different window managers to customize window behavior and aesthetics. 
\subparagraph{}
Themes and icon packs allow for visual overhauls, while wallpapers and desktop widgets add a personal touch to the desktop. Keyboard shortcuts and shell customizations let you streamline tasks and enhance productivity, and you can further modify file managers, terminal emulators, and application launchers to suit your preferences. For those who enjoy automation, scripting with Bash or Zsh allows you to create custom solutions. 


\begin{figure}
\includegraphics[width=0.7\textwidth]{file-3-dtl.jpeg}
\caption{Linux Distros}
\label{fig:Linux Distros}
\end{figure}



\subparagraph{}
System fonts, package manager configurations, and system sounds can all be adjusted to create a unique Linux experience. Additionally, Linux's open-source nature and community support mean that the possibilities for customization are virtually limitless, making it an ideal choice for users who want to tailor their operating system to their specific needs and style.


\subparagraph{User Interface Customization}
Users can personalize their desktop experience by customizing icons, widgets, wallpapers, and more. The Linux community often develops and shares themes and icon packs to enhance aesthetics.

\subparagraph{Kernel and System Optimization}
Advanced users and system administrators can optimize the Linux kernel and system settings to maximize performance and resource utilization for specific hardware and use cases.


\subsection{SUMMARY}
Linux provides extensive customization options, allowing users to personalize their desktop environments, themes, keyboard shortcuts, and even automate tasks using scripting.
\paragraph{}
With a variety of desktop environments and window managers to choose from, users can craft a unique computing experience.
\paragraph{}
Linux's open-source nature and community support enable limitless customization possibilities, making it an ideal choice for those seeking a tailored and flexible operating system.

\begin{table}[H]
\centering
\begin{tabular}{|m{0.2\linewidth}|m{0.35\linewidth}|m{0.15\linewidth}|m{0.2\linewidth}|}
\hline
\textbf{Category} & \textbf{Advantage} & \textbf{Description} & \textbf{Example} \\
\hline
Open Source & Community-driven Development & Collaborative development based on user feedback. & Ubuntu, Fedora \\
\hline
Customizability & Extensive Personalization & Wide variety of themes and software packages. & GNOME, KDE \\
\hline
Stability & Reliable Performance & Long uptime, ideal for critical systems. & CentOS, Debian \\
\hline
Security & Robust Security Measures & Multi-user architecture, frequent security updates. & SELinux, AppArmor \\
\hline
Community Support & Active User Community & Extensive support through forums and websites. & Stack Exchange, LinuxQuestions.org \\
\hline
Resource Efficiency & Lightweight Distributions & Suitable for older hardware. & Lubuntu, Puppy Linux \\
\hline
Software Management & Centralized Package Management & Easy installation and updates. & apt, yum \\
\hline
Compatibility & Broad Hardware Support & Versatile, supports various devices. & Kernel modules \\
\hline
Command Line Interface & Powerful CLI & Efficient task execution and automation. & Bash, Terminal \\
\hline
Innovation & Cutting-edge Technology & Foundation for modern applications. & Docker, Kubernetes \\
\hline
\end{tabular}
\caption{Key Advantages of Linux Operating System}
\label{tab:linux_advantages}
\end{table}


\section{EQUATION}
Algebraic Formulas:

\begin{align}
	&({\alpha}+{\beta})^2 = {\alpha}^2 + {\beta}^2 + 2{\alpha}{\beta}\\
	&{\gamma}^2 - {\lambda}^2 = ({\gamma}+{\lambda})({\gamma}-{\lambda}) \\
	&({\alpha}-{\beta})^3 = {\alpha}^3 - {\beta}^3 - 3{\alpha}{\beta}({\alpha}-{\beta})\\
	&{\omega}^3 - {\delta}^3 = ({\omega}-{\delta})({\omega}^2+ ({\delta}^2+{\omega}{\delta})) \\
	&(a-b)^2 = a^2 - 2ab + b^2 \\
	&({\eta}-{\theta})^2 = ({\eta}^2 + {\theta}^2 - 2({\eta}{\theta})) \\
	%&(a+b)^3 = a^3 + b^3 + 3ab(a+b) \\
	%&a^3 + b^3 = (a+b)(a62-ab+b^2) \\
	%&(a + b)^4 = a^4 + 4a^3b + 6a^2b^2 + 4ab^3 + b^4\\
	&(a – b)^4 = a^4 – 4a^3b + 6a^2b^2 – 4ab^3 + b^4\\
	&a^4 – b^4 = (a – b)(a + b)(a^2 + b^2)\\
	&a^5 – b^5 = (a – b)(a^4 + a^3b + a^2b^2 + ab^3 + b^4)\\
	&(a +b+ c)^2=a^2+b^2+c^2+2ab+2bc+2ca\\
	&({\rho}+ {\alpha}+ {\omega})^2={\rho}^2+{\alpha}^2+{\omega}^2+2{\rho}{\omega}+2{\rho}{\alpha}+2{\rho}{\omega}\\
	&x^3 + y^3 + z^3 - 3xyz = (x + y + z)(x^2 + y^2 + z^2 - xy - yz - xz)
\end{align}


\subsection{PYTHAGOREAN THEOREM}
\[
c^2 = a^2 + b^2
\]
\begin{figure}
\includegraphics[width=0.7\textwidth]{file-11-dtl.jpeg}
\caption{Pythagorean Theorem}
\label{fig:Pythagorean Diagram}
\end{figure}

\subsubsection{Example}
Lets consider a quadratic equation of degree 2 with real roots:

$x^2-x-20=0$

$x^2-5x+4x-20=0$

$x(x-5)+4(x-5)=0$

$(x-5)(x+4)=0$
So, 
$(x-5) =0$, and $(x+4) =0$

imples:
$x-5 = 0$

$x = 5$

And $ x+4 = 0$

$x =-4$

Hence:
The roots of the given quadratic equation are:
$x=5$ and $ x=-4$

%Content page 5
\section{INTEGRATION TABLE}

\subsection{Integration of constant}
\begin{equation*}
	\int k  dx = kx + C
\end{equation*}

\subsection{Closed Curve Integral}
\begin{equation*}
	\oint_C e^x dx = e^x + C
\end{equation*}

\subsection{Tripple Integral}
\begin{equation*}
	\iiint \cos(x,y,z) dx dy dz = \sin(x,y,z) + C
\end{equation*}

\subsection{Definite Integral}
\begin{equation*}
	\int_0^{\pi} \sec^2(x) dx = \tan(x) + C
\end{equation*}

\subsection{Integeration of 1/x}
\begin{equation*}
	\int \frac{1}{x} dx = \ln |x| + C
\end{equation*}


\section{Matrix}

General Matrix : 



\[A=
\begin{bmatrix}
	a & b & \cdots & \cdots & e\\
	f & g & \cdots & \cdots & j\\
	\vdots & \vdots & \ddots & n & \vdots\\
	p & q & r & s & t\\
\end{bmatrix}
\]

Q.Find A+B, where A and B are square matrices given as:
\[A=
\begin{bmatrix}
        0 & -1 & 8\\
        6 & -14 & 2\\
        9 & 5 & 1\\
\end{bmatrix}
\]


\[B=
\begin{bmatrix}
	-5 & 2 & 0\\
	7 & -3 & 4\\
	-1 & 3 &2\\
\end{bmatrix}
\]
OPTIONS:
(a) 
\[
	\begin{pmatrix}
		1 & 0 & 0\\
		-12 & 15 & 0\\
		2 & 5 & 12\\
	\end{pmatrix}
\]
(b)
\[
        \begin{matrix}
		21 & -16 & -1\\
		-12 & 5 & 0\\  
        \end{matrix}
\]
(c)
\[
        \begin{pmatrix}
		-5 & 1 & 8\\
		13 & -17 & 6\\
		8 & 8 & 3\\
        \end{pmatrix}
\]
(d)
\[
        \begin{vmatrix}
		2 & -5 & 7\\
		-12 & 3 & 0\\
		-6 & 1 & 0\\
        \end{vmatrix}
\]

\section{EINSTEIN'S EQUATION}

\begin{figure}
\includegraphics[width=0.7\textwidth]{file-1-dtl.jpeg}
\caption{Einstein's Equation}
\label{fig:Einstein's Equation}
\end{figure}

\subsection{Equation-1}
\begin{equation}
	E=mc^2
\end{equation}

\subsection{Equation-2}

\begin{equation}
	E=h\nu
\end{equation}

\subsection{Equation-3}

\begin{equation}
	v=\frac{c}{\lambda}
\end{equation}

\subsection{Equation-4}

\begin{equation}
	p=mv
\end{equation}

\subsection{Equation-5}

\begin{equation}
	E = \frac{3KT}{2}
\end{equation}

From (1) (2) (3) (4) and (5) we get :

\begin{equation*}
	\lambda = \frac{h}{p} = \frac{h}{mv} = \frac{h}{\sqrt{3mKT}}
\end{equation*}

\section{QUESTIONS}	
\subsection{Question-1}
Q. Solve :
\begin{align}
	(a) \tan(x) \frac{dy}{dx} = y , y(\pi / 2) = \pi/2\\
	(b) (1+\ln(xy))dx + (1+\frac{x}{y})dy=0
\end{align}

\subsection{Question-2}
Q. Solve:
\begin{align*}
	(a) \int_{-2}^{2} \sqrt{1+3x-x^2} dx\\
	(b) \iint_{D} \frac{xe^2}{4-y} dy dx\\
	(c) \iiint_{D} x \sqrt{y^2 - \log{x^2}} dx dy dz
\end{align*}

\subsection{Question-3}
Q. Find determinant:
\[
\begin{vmatrix}5 & 0 & 2 & 0 & -3 \\
	0 & 5 & 0 & 12 & 5 \\
	0 & 0 & 1 & 3 & 0 \\
	1 & 1 & 9 & 1 & 1 \\
	4 & 0 & -2 & 5 & 0 
\end{vmatrix}
\]

\end{document}

