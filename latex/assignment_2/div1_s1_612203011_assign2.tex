\documentclass{article}	
\usepackage{amsmath}

\begin{document}

% Title Page
\title{Why Linux Is Better!}
\author{Arnav Prasad , MIS : 612203011}
\date{\today}
\maketitle

\pagenumbering{gobble}

\newpage

%  Page 2
\pagenumbering{arabic}
\section{OVERVIEW}

Linux is an open-source operating system kernel that serves as the core foundation for various Linux distributions (like Ubuntu, CentOS, and Debian). It provides the essential software components and services necessary to manage hardware resources and run software applications on a computer. 
\paragraph{}
Linux is known for its stability, security, and flexibility and is widely used in servers, desktop computers, embedded systems, and more.

\subsection{OPEN SOURCE AND FREE}
 Most Linux distributions are open source, which means their source code is freely available. This allows for transparency, collaboration, and customization. Additionally, Linux distributions are often available for free, which can be appealing for individuals and organizations with budget constraints.

\subparagraph{Open Source Community}
The vibrant open-source community continuously contributes to improving Linux's performance. With collaboration and testing from developers worldwide, Linux remains at the forefront of performance optimization.

\subparagraph{File Managers}
Linux file managers often allow for customization of file handling, views, and shortcuts, improving file organization and navigation.

\subsection{PERFORMANCE}
Linux is often praised for its efficient resource utilization, making it well-suited for older hardware and resource-constrained environments. It also supports a wide range of hardware architectures.

\subsection{Efficient Resource Utilization}
Linux is known for its efficient use of system resources. It's designed to make the most of available CPU, RAM, and disk space, making it a great choice for both older hardware and resource-constrained environments.

\paragraph{Stability and  Reliability}
Linux is known for its stability and reliability. Many servers and critical infrastructure systems run on Linux due to its ability to run for extended periods without requiring frequent reboots.

\subsection{SECURITY}
 Linux's architecture and security model contribute to its reputation for being more secure than some other operating systems. The community's quick response to security vulnerabilities and the ability to fine-tune security settings are considered advantages.


\subsection{DEVELOPMENT AND SCRIPTING}
 Linux is a popular choice for software development and scripting. Many programming tools, libraries, and frameworks are readily available for Linux, making it a preferred platform for developers.

\subsection{COMMUNITY AND SUPPORT}
 The Linux community is active and passionate, offering extensive online resources, forums, and documentation. Users can easily find help and solutions to their problems.

\paragraph{Privacy}
Linux distributions generally prioritize user privacy and data protection. Users have more control over what data is collected and shared.

\subparagraph{Licensing}
The licensing of Linux and open-source software in general allows users to modify and distribute the software without the same restrictions as proprietary software.

\subsection{COMMAND LINE POWER}
 Linux provides a powerful command-line interface (CLI) that allows users to perform a wide range of tasks efficiently. The terminal provides access to a vast ecosystem of command-line tools and utilities for system administration, scripting, and automation.

\subsection{NO FORCED UPDATES}
Unlike some other operating systems, Linux distributions typically don't force updates on users. You have more control over when and how you update your system.

\subsection{CUSTOMIZABILITY}
Linux offers a high degree of customization. Users can choose from a variety of desktop environments, themes, and software components to create a personalized computing environment that suits their needs.
\newpage
\pagenumbering{roman}
\paragraph{Compatibility and Portability}
Linux is highly portable and can run on various hardware architectures. It's commonly used in embedded systems, mobile devices, and IoT (Internet of Things) devices.

\subparagraph{}
Linux Mint offers various C++ Integrated Development Environments (IDEs) to choose from, and you can install them based on your preferences. 

\paragraph{CUSTOMIZATIONS}
Linux offers extensive customization options that empower users to personalize their computing experience. You can start by choosing from a variety of desktop environments, each with its own look and feel, or opt for different window managers to customize window behavior and aesthetics. 
\subparagraph{}
Themes and icon packs allow for visual overhauls, while wallpapers and desktop widgets add a personal touch to the desktop. Keyboard shortcuts and shell customizations let you streamline tasks and enhance productivity, and you can further modify file managers, terminal emulators, and application launchers to suit your preferences. For those who enjoy automation, scripting with Bash or Zsh allows you to create custom solutions. 
\subparagraph{}
System fonts, package manager configurations, and system sounds can all be adjusted to create a unique Linux experience. Additionally, Linux's open-source nature and community support mean that the possibilities for customization are virtually limitless, making it an ideal choice for users who want to tailor their operating system to their specific needs and style.

\subparagraph{User Interface Customization}
Users can personalize their desktop experience by customizing icons, widgets, wallpapers, and more. The Linux community often develops and shares themes and icon packs to enhance aesthetics.

\subparagraph{Kernel and System Optimization}
Advanced users and system administrators can optimize the Linux kernel and system settings to maximize performance and resource utilization for specific hardware and use cases.


\subsection{SUMMARY}
Linux provides extensive customization options, allowing users to personalize their desktop environments, themes, keyboard shortcuts, and even automate tasks using scripting.
\paragraph{}
With a variety of desktop environments and window managers to choose from, users can craft a unique computing experience.
\paragraph{}
Linux's open-source nature and community support enable limitless customization possibilities, making it an ideal choice for those seeking a tailored and flexible operating system.
\newpage

% Content Page 4
\section{EQUATION}
Algebraic Formulas:

\begin{align}
	&(a+b)^2 = a^2 + 2ab + b^2\\ 
	&a^2 - b^2 = (a+b)(a-b) \\
	&(a-b)^3 = a^3 - b^3 -3ab(a-b) \\
	&a^3-b^3 =  (a-b)(a^2+ab+b^2) \\
	&(a-b)^2 = a^2 - 2ab + b^2 \\
	&(a+b)^3 = a^3 + b^3 + 3ab(a+b) \\
	&a^3 + b^3 = (a+b)(a62-ab+b^2) \\
	&(a + b)^4 = a^4 + 4a^3b + 6a^2b^2 + 4ab^3 + b^4\\
	&(a – b)^4 = a^4 – 4a^3b + 6a^2b^2 – 4ab^3 + b^4\\
	&a^4 – b^4 = (a – b)(a + b)(a^2 + b^2)\\
	&a^5 – b^5 = (a – b)(a^4 + a^3b + a^2b^2 + ab^3 + b^4)\\
	&(a +b+ c)^2=a^2+b^2+c^2+2ab+2bc+2ca\\
	&(x+ y+ z)^2=x^2+y^2+z^2+2xy+2yz+2xz\\
	&x^3 + y^3 + z^3 - 3xyz = (x + y + z)(x^2 + y^2 + z^2 - xy - yz - xz)
\end{align}


\subsection{PYTHAGOREAN THEOREM}
\[
c^2 = a^2 + b^2
\]

\subsubsection{Example}
Lets consider a quadratic equation of degree 2 with real roots:

$x^2-x-20=0$

$x^2-5x+4x-20=0$

$x(x-5)+4(x-5)=0$

$(x-5)(x+4)=0$
So, 
$(x-5) =0$, and $(x+4) =0$

imples:
$x-5 = 0$

$x = 5$

And $ x+4 = 0$

$x =-4$

Hence:
The roots of the given quadratic equation are:
$x=5$ and $ x=-4$

Another example:
$x^2-4=0$

$(x-2)(x+2)=0$

So, 
$(x-2) =0$, and $(x+2) =0$

imples:
$x-2 = 0$

$x = 2$

And $ x+2 = 0$

$x =-2$

Hence:
The roots of the given quadratic equation are:
$x=2$ and $ x=-2$

%Content page 5
\section{INTEGRATION TABLE}

\begin{equation*}
	\int k  dx = kx + C
\end{equation*}

\begin{equation*}
	\int e^x dx = e^x + C
\end{equation*}

\begin{equation*}
	\int \cos(x) dx = \sin(x) + C
\end{equation*}
\begin{equation*}
	\int \sec^2(x) dx = \tan(x) + C
\end{equation*}
\begin{equation*}
	\int \frac{1}{x} dx = \ln |x| + C
\end{equation*}

General Matrix : 
\[A=
\begin{bmatrix}
	a & b & \cdots & \cdots & e\\
	f & g & \cdots & \cdots & j\\
	\vdots & \vdots & \ddots & n & \vdots\\
	p & q & r & s & t\\
\end{bmatrix}
\]

Q.Find A+B, where A and B are square matrices given as:
\[A=
\begin{bmatrix}
        0 & -1 & 8\\
        6 & -14 & 2\\
        9 & 5 & 1\\
\end{bmatrix}
\]


\[B=
\begin{bmatrix}
	-5 & 2 & 0\\
	7 & -3 & 4\\
	-1 & 3 &2\\
\end{bmatrix}
\]
OPTIONS:
(a) 
\[
	\begin{bmatrix}
		1 & 0 & 0\\
		-12 & 15 & 0\\
		2 & 5 & 12\\
	\end{bmatrix}
\]
(b)
\[
        \begin{bmatrix}
		21 & -16 & -1\\
		-12 & 5 & 0\\  
        \end{bmatrix}
\]
(c)
\[
        \begin{bmatrix}
		-5 & 1 & 8\\
		13 & -17 & 6\\
		8 & 8 & 3\\
        \end{bmatrix}
\]
(d)
\[
        \begin{bmatrix}
		2 & -5 & 7\\
		-12 & 3 & 0\\
		-6 & 1 & 0\\
        \end{bmatrix}
\]

\section{EINSTEIN'S EQUATION}
\begin{equation}
	E=mc^2
\end{equation}

\begin{equation}
	E=h\nu
\end{equation}

\begin{equation}
	v=\frac{c}{\lambda}
\end{equation}

\begin{equation}
	p=mv
\end{equation}

\begin{equation}
	E = \frac{3KT}{2}
\end{equation}

From (1) (2) (3) (4) and (5) we get :

\begin{equation*}
	\lambda = \frac{h}{p} = \frac{h}{mv} = \frac{h}{\sqrt{3mKT}}
\end{equation*}

Q. Solve :
\begin{align}
	(a) \tan(x) \frac{dy}{dx} = y , y(\pi / 2) = \pi/2\\
	(b) (1+\ln(xy))dx + (1+\frac{x}{y})dy=0
\end{align}

Q. Solve:
\begin{align*}
	(a) \int_{-2}^{2} \sqrt{1+3x-x^2} dx\\
	(b) \iint_{D} \frac{xe^2}{4-y} dy dx\\
	(c) \iiint_{D} x \sqrt{y^2 - \log{x^2}} dx dy dz
\end{align*}

Q. Find determinant:
\[
\begin{vmatrix}5 & 0 & 2 & 0 & -3 \\
	0 & 5 & 0 & 12 & 5 \\
	0 & 0 & 1 & 3 & 0 \\
	1 & 1 & 9 & 1 & 1 \\
	4 & 0 & -2 & 5 & 0 
\end{vmatrix}
\]

\end{document}

