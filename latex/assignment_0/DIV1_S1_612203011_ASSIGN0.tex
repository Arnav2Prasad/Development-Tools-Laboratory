\documentclass{article}
\usepackage{amsmath} % For mathematical equations

\begin{document}

% Title Page
\title{My First LaTeX Document!}
\author{Arnav Prasad , MIS : 612203011}
\date{\today}
\maketitle

\newpage

% Content Page 2
\section{Why Linux is Better}


\subsection{OPEN SOURCE AND FREE}
 Most Linux distributions are open source, which means their source code is freely available. This allows for transparency, collaboration, and customization. Additionally, Linux distributions are often available for free, which can be appealing for individuals and organizations with budget constraints.

\subsection{CUSTOMIZABILITY}
Linux offers a high degree of customization. Users can choose from a variety of desktop environments, themes, and software components to create a personalized computing environment that suits their needs.

\subsection{PERFORMANCE}
Linux is often praised for its efficient resource utilization, making it well-suited for older hardware and resource-constrained environments. It also supports a wide range of hardware architectures.

\subsection{STABILITY AND RELIABILITY}
Linux is known for its stability and reliability. Many servers and critical infrastructure systems run on Linux due to its ability to run for extended periods without requiring frequent reboots.

\subsection{SECURITY}
 Linux's architecture and security model contribute to its reputation for being more secure than some other operating systems. The community's quick response to security vulnerabilities and the ability to fine-tune security settings are considered advantages.

\subsection{PACKAGE MANAGEMENT}
Linux distributions typically use package managers that allow users to easily install, update, and remove software packages. This centralized approach simplifies software management and reduces the risk of downloading and installing malicious software.
\newpage

% Content Page 3

\subsection{DEVELOPMENT AND SCRIPTING}
 Linux is a popular choice for software development and scripting. Many programming tools, libraries, and frameworks are readily available for Linux, making it a preferred platform for developers.

\subsection{COMMUNITY AND SUPPORT}
 The Linux community is active and passionate, offering extensive online resources, forums, and documentation. Users can easily find help and solutions to their problems.

\subsection{PRIVACY}
Linux distributions generally prioritize user privacy and data protection. Users have more control over what data is collected and shared.

\subsection{LICENSING}
The licensing of Linux and open-source software in general allows users to modify and distribute the software without the same restrictions as proprietary software.

\subsection{COMMAND LINE POWER}
 Linux provides a powerful command-line interface (CLI) that allows users to perform a wide range of tasks efficiently. The terminal provides access to a vast ecosystem of command-line tools and utilities for system administration, scripting, and automation.

\subsection{NO FORCED UPDATES}
Unlike some other operating systems, Linux distributions typically don't force updates on users. You have more control over when and how you update your system.

\subsection{COMPATIBILITY AND PORTABILITY}
Linux is highly portable and can run on various hardware architectures. It's commonly used in embedded systems, mobile devices, and IoT (Internet of Things) devices.


\newpage

% Content Page 4
\section{EQUATION}
An example of a known  mathematical equation:

\subsection{BRIEF}
The theorem was known to ancient civilizations, including the Babylonians and Egyptians, but it is most famously associated with the ancient Greek mathematician Pythagoras, from whom it gets its name.

Overall, the Pythagorean Theorem is a fundamental concept in mathematics with a wide range of applications. It's an essential tool for solving problems involving right triangles and is an important part of geometry and trigonometry.

The Pythagorean Theorem is a fundamental principle in geometry that relates to right triangles. It establishes a relationship between the lengths of the sides of a right triangle, specifically the two shorter sides, known as the legs, and the longest side, known as the hypotenuse.



\subsection{PYTHAGOREAN THEOREM}
The Pythagorean theorem states that for a right triangle, the square of the length of the hypotenuse ($c$) is equal to the sum of the squares of the other two sides ($a$ and $b$):
\[
c^2 = a^2 + b^2
\]


c : represents the length of the hypotenuse.\newline
a and b : represent the lengths of the two legs.

\subsection{QUADRATIC FORMULA}
The solutions of a quadratic equation $ax^2 + bx + c = 0$ are given by:
\[
x = \frac{-b \pm \sqrt{b^2 - 4ac}}{2a}
\]

\subsection{CONVERSE}
The Pythagorean Theorem also has a converse, which states that if the square of the length of the longest side of a triangle is equal to the sum of the squares of the other two sides, then the triangle is a right triangle.\newline

\subsection{APPLICATION}
The Pythagorean Theorem has numerous practical applications in various fields, including geometry, trigonometry, engineering, architecture, and physics. It's used for measuring distances, calculating areas and volumes, and solving problems involving right triangles.

\end{document}

