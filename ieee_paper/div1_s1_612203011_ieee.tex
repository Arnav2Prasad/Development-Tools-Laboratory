\documentclass[conference]{IEEEtran}
\IEEEoverridecommandlockouts

\usepackage{cite}
\usepackage{amsmath,amssymb,amsfonts}
\usepackage{algorithmic}
\usepackage{graphicx}
\usepackage{textcomp}
\usepackage{xcolor}
\def\BibTeX{{\rm B\kern-.05em{\sc i\kern-.025em b}\kern-.08em
    T\kern-.1667em\lower.7ex\hbox{E}\kern-.125emX}}
\begin{document}


\title{Artificial Intelligence Enabled Radio Propagation
for Communications—Part I: Channel
Characterization and Antenna-Channel Optimization\\
\thanks{COEP Tech University .}
}

\author{\IEEEauthorblockN{Arnav Prasad}
\IEEEauthorblockA{\textit{Computer Engineering, College of Engineering, Pune} \\
\textit{Autonomous}\\
Pune, India \\
arnav.prasad.ap@gmail.com}
\and
\IEEEauthorblockN{Arnav Prasad}
\IEEEauthorblockA{\textit{Computer Engineering, College of Engineering, Pune} \\
\textit{Autonomous}\\
Pune, India \\
arnav.prasad.ap@gmail.com}

 }

\maketitle




\section{abstract}
To provide higher data rates, as well as better
coverage, cost efficiency, security, adaptability, and scalability,
the 5G and beyond 5G networks are developed with various
artificial intelligence (AI) techniques. In this two-part article,
we investigate\cite{w11} the application of AI and, in particular, machine
learning (ML) to the study of wireless propagation channels.
It first provides a comprehensive overview of ML for channel
characterization and ML-based antenna–channel optimization in
this first part, and then, it gives a state-of-the-art literature review
of channel scenario identification and channel modeling in Part
II. Fundamental results and key concepts of ML for communi-
cation networks are presented, and widely used ML methods for
channel data processing, propagation channel estimation, and
characterization are analyzed and compared. A discussion of
challenges and future research directions for ML-enabled next-
generation networks of the topics covered in this part rounds off
this article.




\begin{IEEEkeywords}
Human Resource, Competence, Training Programs, APQP ( Advanced Product Quality 
Planning) , Information Management, QMS (Quality Management System), Skill Analysis, Learning Management, Customer Relationship Management (CRM), Database 
Management
\end{IEEEkeywords}

\section{Introduction}
      
HE dramatic increase of the numbers of wireless users
and wireless applications brings new demand and chal-
lenges for wireless comm\cite{222}unication networks. The 5G and
beyond 5G (B5G) networks are expected to provide higher
data rates, as well as better coverage, cost efficiency, security,
adaptability, and scalability. Since 2020, 5G commu-
nication has begun to be deployed worldwide, whereas
studies of sixth-generation (6G) wireless communication net-
works have started in academic and industrial research labs
to further enhance MBB, expand the application and coverage
of the Internet of Things (IoT), and make networks/devices
more intelligent. These new application scenarios give the
6G network a series of new performance requirements:
10–100 million devices connections\cite{666} with the peak data rate
of 1–10 TB/s; the mobility that needs to be supported rises to
higher than 1000 km/h to accommodate ultrahigh-speed train
(uHST), unmanned aerial vehicle (UAV), and satellites;
latencies need to be reduced to fractions of 1 ms to account
for tactile Internet and other real-time control applications; and
reliability of five or even seven times has to be achieved for
mission-critical applications. Also, to provide global coverage,
6G wireless networks will expand from terrestrial communi-
cation networks to space–air–ground–sea integrated networks.
The study of propagation channels is a fundamental aspect
of any wireless communication system design, network
optimization, and performance evaluation. Therefore,
to realize 6G networks to meet the requirements above,
the corresponding wireless channels need to be thoroughly
studied. However, the massive—in terms of number of
devices, number of antennas, bandwidth, and so on—
scenarios not only pose a challenge in performing dedicated
measurement campaigns but also lead to massive amounts of
data that need to be processed and analyzed . Classical
techniques for such analysis, e.g., parameter estimation,
tracking, clustering, and characterization, are generally less
suited for such large amounts of data, either because of the
resulting overhead or because they might miss important
relationships within the data.
On the other hand, artificial intelligence (AI) has been
developed to “simulate the human intelligence\cite{100} processes
by machines, especially computer systems” . Machine
learning (ML) is a branch of AI that enables machines to learn
from a massive amount of data and make decisions and/or
perform actions accordingly without being given any specific
commands. With the help of continually increasing computing
power, ML techniques have achieved great success in big
data processing for many applications, e.g., image processing,
natural language processing, and data mining. Consequently,
ML techniques have also been widely applied to various
problems in communications networks and are expected
to be an integral part of next-generation communication
networks.


\section{AI-/ML-A SSISTED O PTIMIZATION AND M ODELING FOR A NTENNA D ESIGN TO I MPROVE R ADIO P ROPAGATION}
he simplest method to use channel information to adapt
antenna properties for improved antenna–channel interaction
is to rely on an antenna array to provide different spatial
properties of the antenna system. Specifically, one can select
between a finite set of spatial filters, made possible by either
turning on/off different antenna elements (in the case of trans-
mit antenna selection (TAS) ,fixed beams,
or antenna tilt, selecting different preformed beams covering
the horizontal plane, or steering the beam in elevation.
In a generic sense, this approach is about connecting the
antenna array outputs to the transmitter or receiver through
a beamforming network, where the array weights can be set
according to the desired selection functionality.

\paragraph{}
Among these applications, TAS has received the most
attention in the literature , due to the topic being
of interest since the early days of multiantenna systems.
The basic idea of TAS is to devise an algorithm to select P
out of MT transmitting antennas (P < MT ) to be connected
to the RF chains such that the best possible end-to-end link
performance is achieved. This technique is motivated both by
the reduced hardware cost due to the need for fewer RF chains
and the modest loss of performance under some conditions.


\paragraph{}
The interest in applying ML to TAS is not limited to single-
user MIMO, but it has also been considered for multiuser
massive MIMO systems . In , the self-supervised
learning-based Monte Carlo tree search (MCTS) method is
proposed to solve the antenna selection problem using channel
capacity as the key performance indicator. The components
in the TAS system model are mapped to the basic elements
of MCTS (action, tree state, and reward). Linear regression
is used to obtain channel features and provide prediction
to MCTS, facilitating the self-supervised learning process.
Simulation results show high search efficiency with near-
optimal performance, with the BER performance giving a 1 dB
gain over the greedy search selection method. The proposed
method also achieves a similar near-optimal BER performance
as the search-based branch and bound (BAB) method,
but with 50




%%%%%
\section{Computer Vision-Based Cluster Identification}
Computer Vision-Based Cluster Identification: Before
automatic cluster identification methods were devised, MPC
clusters were identified by human inspection as in,
\cite{111}. Despite the somewhat subjective operation model
in human-eyeball clustering, there are still some basic criteria
that are generally applied, including the shape of the potential
cluster, the distribution pattern of the MPCs’ delay and angle,
and the power distribution of all MPCs. All these principles
are visual-based, and hence, it is also possible to use an image
processing method to recognize the MPC cluster. Inspired
by this, some studies focus on computer vision-based cluster
identification. With the consideration of the delay behaviors


\begin{table}[h]
\centering
\begin{tabular}{|c|c|c|c|} % Specify the number of columns and their alignment
\hline
\textbf{Category} & \textbf{Typical Algorithm} & \textbf{Parameter Type} & \textbf{Existing Works} \\ % Table header
\hline
	Traditional Mehod & Barlett A & PAS & [62] \\ % Table content
	Traditional Mehod & ESPRIT & Signal angle & [63] \\ % Table content
	Traditional Mehod & SAGE & MPC Joint & [64] \\ % Table content
	Traditional Mehod& RiMAX & DMC & [65] \\ % Table content
	Traditional Mehod& SVM & Path Loss & [66] \\ % Table content
	ML-based method & ANN & Channel escess & [67] \\ % Table content
	ML-based method & SVM-PCA & AoA and ASA & [68] \\ % Table content
	 ML-based method&Bayesian Learning & DoA & [69] \\ % Table content
	ML-based method & RVM C & DoA & [70] \\ % Table content
	ML-based method & EKF & MPC Joint & [71] \\ % Table content
\hline
\end{tabular}
\caption{Summary of Channel Parameter Estimation}
\label{tab:example}
\end{table}

\begin{figure}[h]
    \centering
    \includegraphics[width=0.5\textwidth]{ieee-2.jpeg}
    \caption{Example Image}
    \label{fig:example}
\end{figure}




of the MPCs, a Hough transform-based clustering algorithm
is present in for vehicle-to-vehicle (V2V) channels,
which exploits the Hough transform \cite{w11} to recognize the
trajectory of MPC in the delay domain and merges the
recognized trajectory into clusters. A PAS-based clustering
and tracking (PASCT) algorithm without any high-resolution
parameter extraction is proposed in; it introduces
the maximum-between-class-variance method  to sepa-
rate the potential cluster groups from the background noise and
further divides the clusters by using the density-peak-search
method . The PASCT algorithm identifies the clusters
directly from the PAS, which can be fast obtained by applying
the Bartlett beamformer. A similar method is also adopted
in , where the cluster is recognized from the PAS by
using image denoising, coarse-grained segmentation, and fine-
grained segmentation.
The vision-based cluster identification follows an intuitive
approach and thus can provide identification results that
conform to human observation and benefit from the rapid
development of computer vision science.

\begin{figure}[h]
    \centering
    \includegraphics[width=0.5\textwidth]{ieee-3.jpeg}
    \caption{Illustration of the evolving-based cluster identification}
    \label{fig:example}
\end{figure}



In summary, the ML-based cluster identification solutions
show a significant ability to identify the MPC clusters for
further analyzing and modeling, and each identification solu-
tion has its own advantages and limitations. Considering that
there is no identification ground truth of the measurement data,
how to properly quantify the performance of the identification
method is still a challenging issue. \cite{66}One way is by evaluating
the identification method by using synthetic data generated
from a cluster-based channel model, where the identification
ground truth of the synthetic data can be easily acquired.
Another solution is to use the statistical figure of merits
for data clustering algorithms, e.g., the Calinski–Harabasz
index , generalized Dunn’s (GD) index, Xie–Beni
(XB) index , or Davies–Bouldin index . Several
measuring methods, including the indices above, are compared
in , where the XB index and GD index generally show
good performance for evaluating the MPC cluster identifi-
cation result. Table III summarizes the existing clustering
methods for wireless channels.




\section{ML-Based Beam Selection and Antenna Tilt Optimization}
Using ML to determine antenna tilt for coverage and
interference optimization is also a popular subject in the
literature . In simple terms, a suitable antenna
tilt can improve signal reception within a cell and reduce
interference toward other cells, leading to a higher signal-to-
interference-plus-noise ratio (SINR) received by the users and
increased sum data rate in the network. However, the
traditional fixed-tilt strategy is not adequate for the complex
coverage and interference problem in heterogeneous networks
(HetNets). In \cite{55} a distributed reinforcement learning
algorithm is proposed, which does not need a base station or
network-wide knowledge of hotspot locations. In the simula-
tion results in that paper, the Boltzmann exploration algorithm
can achieve convergence to a near-optimal solution within
limited iterations and improve the throughput fairness by 45
56 fixed strategies. It is shown in that distributed rein-
forcement learning is also attractive for the antenna tilting
for self-optimization of the RAN, even for homogeneous
networks. Simulation results show a 30
rate in an urban scenario when the tilt angle is optimized


\section{Data-Driven Design of Antenna Patterns}
To benefit more from channel information than aiding in
the selection of antenna/beam/tilt, the channel data can be
more directly utilized in optimizing antenna–channel inter-
action to improve system performance. Since \cite{44}optimizing
For cluster recognition/extraction, most of the existing
works rely on unsupervised clustering algorithms, e.g.,
K-means or fuzzy-C-means. However, the unsupervised algo-
rithms generally rely on preset parameters, e.g., the number
and the position of initial cluster-centroids. Thus, the current
clustering algorithm requires different presettings for different
channel data, which requires extensive manual adjustment to
maintain the clustering accuracy for nonstationary channels.
On the other hand, the ANN-based DL shows great flexibility
for the applications of target recognition and has already been
extended to solve the clustering problem , which
is highly related to the MPC’s cluster recognition. Neverthe-
less, the accuracy of the DL-based cluster recognition is not
increased as expected compared to the existing unsupervised
clustering methods. Therefore, it requires more studies on
how to further improve the accuracy and efficiency of the
DL-based clustering methods. At the same time, the possibility
of tracking joint clustering of time-varying MPCs also requires
further investigation.antenna–channel interaction is mainly about “far-field match-
ing,” one can design the antenna pattern for some desired
properties


\section{DL-Based Cluster Identification}
For cluster recognition/extraction,\cite{33} most of the existing
works rely on unsupervised clustering algorithms, e.g.,
K-means or fuzzy-C-means. However, the unsupervised algo-
rithms generally rely on preset parameters, e.g., the number
and the position of initial cluster-centroids. Thus, the current
clustering algorithm requires different presettings for different
channel data, which requires extensive manual adjustment to
maintain the clustering accuracy for nonstationary channels.
On the other hand, the ANN-based DL shows great flexibility
for the applications of target recognition and has already been
extended to solve the clustering problem , which
is highly related to the MPC’s cluster recognition. Neverthe-
less, the accuracy of the DL-based cluster recognition is not
increased as expected compared to the existing unsupervised
clustering methods. Therefore, it requires more studies on
how to further improve the accuracy and efficiency of the
DL-based clustering methods. At the same time, the possibility
of tracking joint clustering of time-varying MPCs also requires
further investigation.


\begin{figure}[h]
    \centering
    \includegraphics[width=0.5\textwidth]{ieee-1.jpeg}
    \caption{Example Image}
    \label{fig:example}
\end{figure}

\section{AI-Based Applications for mmWave/TH z-Band Communications}
Channel measurements\cite{22} (especially in mmWave and THz
band) are often accompanied by maddening time consumption,
engineering problems, and capital costs. The ML method
needs training data, and reliable training data should come
from the measurements in actual scenarios. In this sense, data
acquisition is the bottleneck of many ML-based applications.
Admittedly, some simulation methods can generate synthetic
training data, but the simulation methods themselves also need
measurement data for evaluation and verification. Hence, con-
ducting sufficient measurement campaigns to support AI-based
applications is one challenging aspect in the future for
mmWave/THz channels. Meanwhile, the mmWave/THz chan-
nels with the ultrawideband and ultramassive MIMO have
shown some new properties, e.g., channel sparsity, chan-
nel hardening, and nonstationarity in time/spatial/frequency
domains. These new channel properties may significantly
affect channel data processing and have not been considered
yet in the existing AI-based applications, e.g., channel sparsity
property may contribute to cluster identification; channel hard-
ening may improve the scenario identification. How to exploit
new channel properties to improve the efficiency, accuracy,
and robustness of AI-based applications in communications
still requires further investigation

\section{CONCLUSION}
AI techniques have become a necessary tool to develop
the next-generation communication network. In this arti-
cle, we provide a comprehensive overview of AI-enabled
data processing for propagation channel studies, includ-
ing channel parameter estimation and characterization and
antenna–channel optimization in Part I, whereas the scenario
identification and channel modeling/prediction are covered
in Part II . This article demonstrates the early results
of the related works and illustrates the typical AI/ML-based
solutions for each topic. Based on the state of the art, the
future challenges of AI/ML-based channel data processing
techniques are given as well.


\bibliographystyle{IEEEtran}  % Choose an appropriate IEEE bibliography style
\bibliography{i}

\end{document}
